\documentclass[12pt]{article}
\usepackage{tikz}
\usetikzlibrary{matrix,arrows}
\usepackage{graphicx}
\usepackage{amsmath} 
\usepackage{amssymb} 
\usepackage{amsthm}
\usepackage{enumitem} % to change appearance of enum and item environments
\usepackage{framed}
\usepackage{color}
\usepackage{multicol}
%\usepackage{yhmath}
% adjust the margins using the geometry package
\usepackage[left=0.75in, right=0.75in, top=0.75in, bottom=0.75in]{geometry}
\usepackage[parfill]{parskip}
 %\usepackage{mathpazo}
%\usepackage{euler}

% customize the headers using the fancyhdr package
\usepackage{fancyhdr}
\pagestyle{fancy}

\usepackage{hyperref}

\usepackage{mathrsfs}
\usepackage{mathtools}

\renewcommand{\headrulewidth}{0.4pt}
\renewcommand{\footrulewidth}{0.4pt}

\newenvironment{hwproblem}[1]{{\large \bfseries Problem #1\,:} \begin{trivlist}\item[]\vspace{-0.5ex}}{\end{trivlist}\vspace{3ex}}
\newenvironment{hwproblemquestion}[1]{{\bfseries Question #1\,:} \begin{trivlist}\item[]\vspace{-1.5ex}}{\end{trivlist}\vspace{3ex}}
\newenvironment{lecexercise}[1]{{\large \bfseries Exercise #1\,:} \begin{trivlist}\item[]\vspace{-0.5ex}}{\end{trivlist}\vspace{3ex}}
\newenvironment{note}[1]{{\large \bfseries Note #1\,:} \begin{trivlist}\item[]\vspace{-0.5ex}}{\end{trivlist}\vspace{3ex}}
\newenvironment{hwsubpart}[1]{{\bfseries #1\,:} \begin{trivlist}\item[]\vspace{-0.5ex}}{\end{trivlist}\vspace{3ex}}


% create theorem style
\newtheoremstyle{break}% name
  {}%         Space above, empty = `usual value'
  {}%         Space below
  {}%         Body font
  {}%         Indent amount (empty = no indent, \parindent = para indent)
  {\bfseries}% Thm head font
  {.}%        Punctuation after thm head
  {\newline}% Space after thm head: \newline = linebreak
  {}%         Thm head spec

\theoremstyle{break}
\newtheorem{hwquestion}{Question}

\newtheorem{theorem}{Theorem}
\numberwithin{theorem}{subsection}

\newtheorem*{lemma*}{Lemma}
\newtheorem{lemma}{Lemma}
\numberwithin{lemma}{subsection}

\newtheorem{corollary}{Corollary}
\numberwithin{corollary}{subsection}


\newtheorem*{definition}{Definition}

\newtheorem*{note*}{Note}

\newtheorem*{remark}{Remark}

\newtheorem*{hint}{Hint}

\newtheorem*{example}{Example}

\numberwithin{equation}{subsection}



% customize enumerate and itemize environments
\setlist[itemize]{labelsep=1ex,itemsep=1.5ex,parsep=0ex,leftmargin=4ex,topsep=0.5ex}
\setlist[enumerate]{labelsep=1ex,itemsep=1.5ex,parsep=0ex,leftmargin=4ex,topsep=0.5ex}




\rhead{Calculus 3}
\lhead{Matthew Stringer} % your name
\lfoot{}
\cfoot{Calculus 3 Notes}  % change to the corresponding number
\rfoot{\thepage}

\setlength{\headheight}{15pt}
\title{Calculus 3 Notes} % change to the corresponding number
\date{}
\author{Matthew Stringer} % your name and ID number

% Anything above the \begin{document} is the template. If you wish to start a new document using this template, erase everything inside of the \begin{document}...\end{document}
\allowdisplaybreaks
\begin{document}
\maketitle
\newpage
\tableofcontents
\newpage

\section{Section 14.6}

\subsection{Standard Linear Approximation}
For $f(x,y)$ at $(x_0, y_0)$, the Standard Linear Approximation of $f(x,y)$ is:
\begin{equation} 
L(x,y) = f(x_0,y_0) + f_x(x_0,y_0)(x-x_0) + f_y(x_0,y_0)(y-y_0) 
\end{equation}

\subsection{The Error in the Standard Linear Approximation}
If $f$ has continuous first and second partial derivative throughout an open set containing a 
rectangle $R$ centered at $(x_0, y_0)$ and if $M$ is any upper bound for the values of $|f_{xx}|,
|f_{yy}|,$ and $|f_{xy}|$ on $R$, then the error $E(x,y)$ incurred in replacing $f(x,y)$ on $R$ by
its linearizion satisfies the inequality
\begin{equation}
|E(x,y)| \leq \frac12 M(|x-x_0| + |y-y_0|)^2 .
\end{equation}

\subsection{Tangent Plane}
For $f(x,y)$ at $(x_0, y_0)$, the tangent plane of $f(x,y)$ is:
\begin{equation} 
f_x(x_0,y_0)(x-x_0) + f_y(x_0,y_0)(y-y_0) 
\end{equation}
If you have a surface $z = f(x,y)$ at $P(x_0,y_0,z_0)$ use 
\begin{equation}
f_x(x_0,y_0)(x-x_0) + f_y(x_0,y_0)(y-y_0) - (z-z_0) = 0
\end{equation}

\subsection{Normal Line}
The normal line to $f(x,y,z)$ at $P_0 (x_0,y_0,z_0)$ has the following equations:
\begin{align*}
x = x_0 + f_x(P_0)t \\
y = y_0 + f_y(P_0)t \\
z = z_0 + f_z(P_0)t \\
\end{align*}

\end{document}