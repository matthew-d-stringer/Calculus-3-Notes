\documentclass[12pt]{article}
\usepackage{tikz}
\usetikzlibrary{matrix,arrows}
\usepackage{graphicx}
\usepackage{amsmath} 
\usepackage{amssymb} 
\usepackage{amsthm}
\usepackage{enumitem} % to change appearance of enum and item environments
\usepackage{framed}
\usepackage{color}
\usepackage{multicol}
\usepackage{siunitx}
%\usepackage{yhmath}
% adjust the margins using the geometry package
\usepackage[left=0.75in, right=0.75in, top=0.75in, bottom=0.75in]{geometry}
\usepackage[parfill]{parskip}
 %\usepackage{mathpazo}
%\usepackage{euler}

% customize the headers using the fancyhdr package
\usepackage{fancyhdr}
\pagestyle{fancy}

\usepackage{hyperref}

\usepackage{mathrsfs}
\usepackage{mathtools}

\renewcommand{\headrulewidth}{0.4pt}
\renewcommand{\footrulewidth}{0.4pt}

\newenvironment{hwproblem}[1]{{\large \bfseries Problem #1\,:} \begin{trivlist}\item[]\vspace{-0.5ex}}{\end{trivlist}\vspace{3ex}}
\newenvironment{hwproblemquestion}[1]{{\bfseries Question #1\,:} \begin{trivlist}\item[]\vspace{-1.5ex}}{\end{trivlist}\vspace{3ex}}
\newenvironment{lecexercise}[1]{{\large \bfseries Exercise #1\,:} \begin{trivlist}\item[]\vspace{-0.5ex}}{\end{trivlist}\vspace{3ex}}
\newenvironment{note}[1]{{\large \bfseries Note #1\,:} \begin{trivlist}\item[]\vspace{-0.5ex}}{\end{trivlist}\vspace{3ex}}
\newenvironment{hwsubpart}[1]{{\bfseries #1\,:} \begin{trivlist}\item[]\vspace{-0.5ex}}{\end{trivlist}\vspace{3ex}}


% create theorem style
\newtheoremstyle{break}% name
  {}%         Space above, empty = `usual value'
  {}%         Space below
  {}%         Body font
  {}%         Indent amount (empty = no indent, \parindent = para indent)
  {\bfseries}% Thm head font
  {.}%        Punctuation after thm head
  {\newline}% Space after thm head: \newline = linebreak
  {}%         Thm head spec

\theoremstyle{break}
\newtheorem{hwquestion}{Question}

\newtheorem{theorem}{Theorem}
\numberwithin{theorem}{subsection}

\newtheorem*{lemma*}{Lemma}
\newtheorem{lemma}{Lemma}
\numberwithin{lemma}{subsection}

\newtheorem{corollary}{Corollary}
\numberwithin{corollary}{subsection}


\newtheorem*{definition}{Definition}

\newtheorem*{note*}{Note}

\newtheorem*{remark}{Remark}

\newtheorem*{hint}{Hint}

\newtheorem*{example}{Example}

\numberwithin{equation}{subsection}



% customize enumerate and itemize environments
\setlist[itemize]{labelsep=1ex,itemsep=1.5ex,parsep=0ex,leftmargin=4ex,topsep=0.5ex}
\setlist[enumerate]{labelsep=1ex,itemsep=1.5ex,parsep=0ex,leftmargin=4ex,topsep=0.5ex}




\rhead{Calculus 3}
\lhead{Matthew Stringer} % your name
\lfoot{}
\cfoot{Calculus 3 Notes}  % change to the corresponding number
\rfoot{\thepage}

\setlength{\headheight}{15pt}
\title{Calculus 3 Notes} % change to the corresponding number
\date{}
\author{Matthew Stringer} % your name and ID number

\newcommand{\ihat}{\hat \i}
\newcommand{\jhat}{\hat \j}
\newcommand{\vect}[1]{\boldsymbol{#1}}

% Anything above the \begin{document} is the template. If you wish to start a new document using this template, erase everything inside of the \begin{document}...\end{document}
\allowdisplaybreaks
\begin{document}
\maketitle
\href{https://ocw.mit.edu/courses/mathematics/18-02sc-multivariable-calculus-fall-2010/index.htm}{\textbf{Course Page}}
\tableofcontents
\newpage

\section{Chapter 12}

\subsection{Component form of a vector}
The component form of $\vec{PQ}$ where both $P$ and $Q$ are defined is 
\begin{equation*}
	Q-P
\end{equation*}

\subsection{Vector projection}
The projection of a vector $u$ onto $v$ is 
\begin{equation*}
	\text{proj}_v u = \left( \frac{u \cdot v}{|v|^2} \right)v
\end{equation*}
The component of the projection is 
\begin{equation*}
	|u|\cos(\theta) = \frac{u \cdot v}{|v|}
\end{equation*}

\subsection{Area of a Parallelogram/Triangle}
Let a parallelogram or triangle is defined with 3 points, $P$,$Q$, and $R$.
Also let $A_p$ be the area of a parallelogram and $A_t$ be the area of a triangle.
\begin{equation*}
	A_p = | \vec{PQ} \times \vec{PR} |
\end{equation*}
\begin{equation*}
	A_t = \frac{A_p}{2} = \frac{|\vec{PQ} \times \vec{PR}|}{2}
\end{equation*}

\subsection{Distance from point to line}
Let $P$ be a reference point on the line, $S$ be a point off of the line, and let $v$ be a vector 
parallel to the line.
Also let the distance between the line and $S$ be $d$.
\begin{equation*}
	d = \frac{|\vec{PS} \times v|}{|v|}
\end{equation*}

\subsection{Angle between 2 vectors}
\begin{equation*}
	\theta = \cos^{-1} \left( \frac{u \cdot v}{|u||v|} \right)
\end{equation*}

\subsection{Angle between 2 planes}
The vector normal to a plane is equivalent to the coefficients of $x$, $y$, and $z$.
Then just find the angle between the two normal vectors

\subsection{Work}
The work done with force vector $F$ and distance vector $d$ is 
\begin{equation*}
	W = F \cdot d
\end{equation*}

\section{Section 13.2}

\subsection{Ideal Projectile Motion}
If $v_0$ makes an angle $\alpha$ with the horizon, then 
\begin{equation*}
v_0 = (|v_0|\cos \alpha){\ihat} + (|v_0|\sin \alpha){\jhat}
\end{equation*}
Also, assume $r_0 = 0{\ihat} + 0{\hat \j}$. With these formulas it can be derived that
\begin{equation*}
r(t) = -\frac12 gt^2 {\hat \j} + v_0t
\end{equation*}

\subsubsection{Range of a Projectile}
\begin{equation*}
R = \frac{v_0^2}{g} \sin (2\alpha)
\end{equation*}
Where $R$ is the range, $v_0$ is the initial velocity, $g$ is acceleration due to gravity, and 
$\alpha$ is the launch angle

\subsubsection{Maximum Height of a Projectile}
Lets say that $y_{max}$ is the maximum height, then:
\begin{equation*}
	y_{max} = \frac{v_0 \sin \alpha}{2g}
\end{equation*}

\section{Section 13.3}

\subsection{Finding Tangent Vector to a Path}
If $r(t)$ is the function for the path, then $r'(t)$ represents the tangent vector for the path.

\subsection{Length of Path}
If $r(t)$ is the function for the path and $D$ represents the length of the path from $t \in (t_0, t_1)$,
then 
\begin{equation*}
	D = \int_{t_0}^{t_1} |r'(t)|dt
\end{equation*}

\section{Section 13.4}

\subsection{Curvature}
If $T$ is the tangent unit vector of a smooth curve, then the curvature of the curve is
\begin{equation*}
\kappa = \left| \frac{dT}{ds} \right|
\end{equation*}
where s is arc length.

\subsection{Formula for Calculating Curvature}
If $s(t)$ is a smooth curve, then curvature is 
\begin{equation*}
%\kappa = \frac1{|s'(t)|} \cdot \left|\frac{dT}{dt}\right|
\kappa = \left| \frac{\frac{dT}{dt}}{\frac{ds}{dt}} \right|
\end{equation*}
where 
\begin{equation*}
T = \frac{s'(t)}{|s'(t)|}
\end{equation*}
and $v = r'(t)$

\subsection{Principal Unit Normal}
At any point $\kappa \neq 0$, the principal unit normal vector for a curve is 
\begin{equation*}
N = \frac{1}{\kappa} \frac{dT}{ds}
\end{equation*}

\subsection{Formula for Calculating N}
If $r(t)$ is a smooth curve, then the principal unit normal is
\begin{equation*}
N = \frac{\frac{dT}{dt}}{\left| \frac{dT}{dt} \right |}
\end{equation*}

\section{Section 13.5}

\subsection{Binormal Vector}
The binormal vector ($B$) is a vector that is orthogonal to both $T$ and $N$. 
\begin{equation*}
B = T \times N 
\end{equation*}

\subsection{Acceleration Vector}
If the acceleration vector is 
\begin{equation*}
a = a_T T + a_N N
\end{equation*}
then 
\begin{equation*}
a_T = s''(t) = \frac{d}{dt} |s'(t)| \text{ and } a_N = \kappa \left( \frac{ds}{dt} \right)^2 = \kappa |s'(t)|^2
\end{equation*}
also
\begin{equation*}
a_N = \sqrt{|s''(t)|^2 - a_T^2}
\end{equation*}

\subsection{Torsion}
\begin{equation*}
\frac{dB}{ds} = \frac{d(T \times N)}{ds} = \frac{dT}{ds} \times N + T \times \frac{dN}{ds}
\end{equation*}
Since $N$ has the same direction as $\frac{dT}{ds}$, $N \times \frac{dT}{ds} = 0$
\begin{equation*}
\frac{dB}{ds} = T \times \frac{dN}{ds}
\end{equation*}
Because $\frac{dB}{ds}$ is orthogonal to $B$ and $T$, it must be parallel to $N$. Therefore, 
$\frac{dB}{ds}$ is a scalar multiple of $N$ and 
\begin{equation*}
\frac{dB}{ds} = -\tau N
\end{equation*}
where $\tau$ is called the \textit{torsion} of the curve. Note that 
\begin{equation*}
\frac{dB}{ds} \cdot N = -\tau N \cdot N = -\tau
\end{equation*}

\subsection{Calculating Torsion}
If $B = T \times N$, the torsion of a smooth curve is 
\begin{equation*}
\tau = - \frac{dB}{ds} \cdot N
\end{equation*}
also, 
\begin{equation*}
\tau = \frac{
	\begin{vmatrix}
	\dot x & \dot y & \dot z \\
	\ddot x & \ddot y & \ddot z \\
	\dddot x & \dddot y & \dddot z \\
	\end{vmatrix}
}{| v \times a |^2}
\end{equation*}

\section{Section 13.6}
\subsection{Motion in Polar and Cylindrical Coordinates}
When a particle at $P(r,\theta)$ moves along a curve in the polar coordinate plane, we express its
position in terms of moving unit vectors:
\begin{equation*}
u_r = (\cos \theta) \ihat + (\sin \theta) \jhat \quad u_\theta = -(\sin \theta)\ihat + (\cos \theta)\jhat
\end{equation*}
where $u_r$ is tangent to the vector $\begin{bmatrix} \theta \\ r \end{bmatrix}$ and $u_\theta$ is 
normal to it. When we differentiate $u_r$ and $u_\theta$ with respect to $t$, we can see how they 
change with time.
\begin{equation*}
\dot u_r = (-\dot{\theta}\sin \theta) \ihat + (\dot{\theta}\cos \theta) \jhat \quad 
u_\theta = -(\dot{\theta}\sin \theta) \ihat - (\dot{\theta}\sin \theta) \jhat
\end{equation*}
which equals 
\begin{equation*}
\dot u_r = \dot{\theta} u_\theta \quad u_\theta = -\dot{\theta} u_r
\end{equation*}

\subsection{Polar Velocity Vector}
\begin{equation*}
v = \dot r = \frac{d}{dt} \left( r u_r \right) = \dot r u_r + r \dot u_r = \dot r u_r + r \dot \theta u_\theta
\end{equation*}

\subsection{Polar Acceleration Vector}
\begin{equation*}
a = \dot v = (\ddot r u_r + \dot r \dot u_r) + 
(\dot r \dot \theta u_\theta + r \ddot \theta u_\theta + r \dot \theta \dot u_\theta)
\end{equation*}
which becomes 
\begin{equation*}
a = (\ddot r - r \dot \theta^2)u_r + (r \ddot \theta + 2 \dot r \dot \theta) u_\theta
\end{equation*}

\subsection{Planet Movement}
If $r$ is the radius vector from the center of a sun of mass $M$ to the center of a planet of mass $m$,
then the force $F$ of the gravitational attraction between the planet and the sun is:
\begin{equation*}
F = - \frac{GmM}{|r|^2} \frac{r}{|r|}
\end{equation*}
where $G$ is the universal gravitational constant. 
\begin{equation*}
G = 6.6726 \times 10^{-11} \si{\newton \meter^2 \kilogram^{-2}}
\end{equation*}
By using Newton's second law, we can find
\begin{equation*}
\ddot r = - \frac{GM}{|r|^2} \frac{r}{|r|}
\end{equation*}
Because $\ddot r$ is a scalar multiple of $r$, we know 
\begin{equation*}
r \times \ddot r = 0
\end{equation*}
so we know
\begin{equation*}
\frac{d}{dt} (r \times \dot r) = \dot r \times \dot r + r \times \ddot r = r \times \ddot r = 0
\end{equation*}
meaning that the cross between $r$ and $\dot r$ is constant
\begin{equation*}
r \times \dot r = C
\end{equation*}

\subsection{Kepler's First Law}
The eccentricity of the ellipse that the planet follows is 
\begin{equation*}
e = \frac{r_0 v_0^2}{GM} - 1
\end{equation*}
and the polar equation is 
\begin{equation*}
r = \frac{(1+e)r_0}{1+e \cos \theta}
\end{equation*}
where $r_0$ is the minimum distance from the sun. The sun's mass $M$ is 
$1.99 \times 10^{30} \si{ \kilogram}$

\subsection{Kepler's Third Law}
The time $T$ it takes a planet to go around its sun once is the planet's orbital period. Kepler's
Third Law says that and the orbit's semimajor axis $a$ are related by
\begin{equation*}
\frac{T^2}{a^3} = \frac{4\pi^2}{GM}
\end{equation*}

\section{Section 14.6}

\subsection{Standard Linear Approximation}
For $f(x,y)$ at $(x_0, y_0)$, the Standard Linear Approximation of $f(x,y)$ is 
\begin{equation} 
L(x,y) = f(x_0,y_0) + f_x(x_0,y_0)(x-x_0) + f_y(x_0,y_0)(y-y_0) 
\end{equation}

\subsection{The Error in the Standard Linear Approximation}
If $f$ has continuous first and second partial derivative throughout an open set containing a 
rectangle $R$ centered at $(x_0, y_0)$ and if $M$ is any upper bound for the values of $|f_{xx}|,
|f_{yy}|,$ and $|f_{xy}|$ on $R$, then the error $E(x,y)$ incurred in replacing $f(x,y)$ on $R$ by
its linearizion satisfies the inequality
\begin{equation}
|E(x,y)| \leq \frac12 M(|x-x_0| + |y-y_0|)^2 .
\end{equation}

\subsection{Tangent Plane}
For $f(x,y)$ at $(x_0, y_0)$, the tangent plane of $f(x,y)$ is:
\begin{equation} 
f_x(x_0,y_0)(x-x_0) + f_y(x_0,y_0)(y-y_0) 
\end{equation}
If you have a surface $z = f(x,y)$ at $P(x_0,y_0,z_0)$ use 
\begin{equation}
f_x(x_0,y_0)(x-x_0) + f_y(x_0,y_0)(y-y_0) - (z-z_0) = 0
\end{equation}

\subsection{Normal Line}
The normal line to $f(x,y,z)$ at $P_0 (x_0,y_0,z_0)$ has the following equations:
\begin{align*}
x = x_0 + f_x(P_0)t \\
y = y_0 + f_y(P_0)t \\
z = z_0 + f_z(P_0)t \\
\end{align*}

\newpage
\section{Section 14.7}

\subsection{Definitions of local maximums and minimums}
If $f(x,y)$ is defined on a region $R$ containing the point $(a,b)$, then:
\begin{enumerate}
\item $f(a,b)$ is a \textbf{local maximum} value of $f$ if $f(a,b) \geq f(x,y)$ for all domain
	points $(x,y)$ in an open disk centered at $(a,b)$.
\item $f(a,b)$ is a \textbf{local minimum} value of $f$ if $f(a,b) \leq f(x,y)$ for all domain
	points $(x,y)$ in an open disk centered at $(a,b)$.
\end{enumerate}

\subsection{First Derivative Test for Local Extreme Values}
If all partial derivatives are equal to zero or undefined, then they are critical points.
In order to find local extrema, you must set all partial derivatives to zero and solve the system
of equations.

\subsection{Second Derivative Test for Local Extreme Values}
Let $D$ be the \textbf{discriminant} or \textbf{Hessian} of $f$ so that
\begin{equation*}
D = f_{xx}f_{yy} - f_{xy}^2
\end{equation*}
then 
\begin{enumerate}
\item $f$ has a \textbf{local maximum} at $(a,b)$ if $f_{xx} < 0$ and $D > 0$ at $(a,b)$.
\item $f$ has a \textbf{local minimum} at $(a,b)$ if $f_{xx} > 0$ and $D > 0$ at $(a,b)$.
\item $f$ has a \textbf{saddle point} at $(a,b)$ if $D < 0$ at $(a,b)$.
\item \textbf{the test is inconclusive} at $(a,b)$ if $D = 0$ at $(a,b)$ and another method must be
	in order to determine the behavior at $(a,b)$.
\end{enumerate}

\subsection{Finding Absolute Maxima and Minima on Closed Bounded Regions}
In order to find absolute extrema for $f(x,y)$ on a closed and bounded region $R$, 
\begin{enumerate}
\item \textit{List the interior points of $R$} where $f$ may have local maxima or minima and 
	evaluate $f$ at these points. These are critical points of $f$.
\item \textit{List the boundary points of $R$} where $f$ has local maxima and minima and evaluate 
	$f$ at these points. For every boundary, fix one or more of the variables in order to create a
	function of a single variable and find its local maxima and minima. 
\item \textit{Look through the lists} for the maximum and minimum values of $f$. These will be the
	absolute maximum and minimum values of $f$ on $R$.
\end{enumerate}

\section{Section 14.8}
Using Lagrange multipliers, you can find extreme values of a function whose domain is constrained to
lie within a subset of a plane.

\subsection{Lagrange Multipliers}
If you have two functions, $f(x,y)$ and $g(x,y) = c$, you can find extreme values of $f$ on $g(x,y) = c$ by finding
locations where $\nabla f = \lambda \nabla g$. $\lambda$ is the Lagrange Multiplier. 

\section{Section 14.9}

\subsection{Local Linearization}
The linearization of a function at the point $(x_0, y_0)$ is 
\begin{equation*}
	L(x,y) = f(x_0, y_0) + f_x(x_0, y_0)(x - x_0) + f_y(x_0, y_0)(y - y_0)
\end{equation*}
If $\vect{x_0} = (x_0, y_0)$ and $\vect{x} = (x,y)$,
\begin{equation*}
	L(\vect{x}) = f(\vect{x_0}) + \nabla f \cdot (\vect{x} - \vect{x_0})
\end{equation*}

\subsection{Quadratic Approximation}
The approximation of a function at the point $(x_0, y_0)$ is 
\begin{align*}
	Q(x,y) = &f(x_0, y_0) + f_x(x_0, y_0)(x-x_0) + f_y(x_0, y_0)(y-y_0) \\
	& + \frac12 f_{xx}(x_0, y_0)(x-x_0)^2 + f_{xy}(x_0, y_0)(x-x_0)(y-y_0) + \frac12 f_{yy}(x_0, y_0)(y-y_0)^2
\end{align*}

\subsubsection{Hessian Matrix}
Let $f$ be a function of $(x,y)$
\begin{equation*}
	\vect{H} = 
	\begin{bmatrix}
		f_{xx} & f_{xy} \\
		f_{yx} & f_{yy} 
	\end{bmatrix}
\end{equation*}

\subsubsection{Representing Quadratic Forms with vectors}
\begin{equation*}
ax^2 + 2bxy + cy^2 = 
\begin{bmatrix}
x & y
\end{bmatrix}
\begin{bmatrix}
a & b \\
b & c
\end{bmatrix}
\begin{bmatrix}
x \\
y
\end{bmatrix}
\end{equation*}
You can also let 
\begin{equation*}
\vect{A} = 
\begin{bmatrix}
a & b \\
b & c
\end{bmatrix}
\end{equation*}
and 
\begin{equation*}
\vect{X} = 
\begin{bmatrix}
x \\
y
\end{bmatrix}
\end{equation*}
to get 
\begin{equation*}
ax^2 + 2bxy + cy^2 = \vect{X}^T \vect{A} \vect{X}
\end{equation*}

\subsubsection{Vector form of Quadratic Approximation}
Let $X = \begin{bmatrix} x \\ y \end{bmatrix} $ and $X_0 = \begin{bmatrix} x_0 \\ y_0 \end{bmatrix} $
\begin{equation*}
	Q(X) = f(X_0) + \nabla f \cdot (X - X_0) + \frac12 (X - X_0)^T \vect{H}_{X_0} (X - X_0)
\end{equation*}


\section{Chapter 15 Double Integral}

\subsection{Definition of the Double Integral}
Double Integral = volume below the graph $z = f(x,y)$ over a region $R$ in xy-plane.
\begin{equation*}
	\int \int_R f(x,y)dA
\end{equation*}
\textbf{Definition: } Cut R into small pieces of area $\Delta A$
Lets divide the region $R$ into $i$ x and y components
The height of each rectangular prism is $f(x_i, y_i)$ and the area of the region underneath is $\Delta A_i$
\begin{equation*}
	\sum_i f(x_i,y_i) \Delta A_i 
\end{equation*}
Finally, you get take the limit as $\Delta A_i \to 0$ to get the double integral.

\subsection{Calculating the Double Integral}
To compute the double integral, take \textbf{slices}. 
Let $S(x)$ = area of slice by plane parallel to the yz-plane.
Then, 
\begin{align*}
	\text{volume} = \int_{x_{min}}^{x_{max}} S(x)dx. \\
	S(x) = \int_{y_{min}(x)}^{y_{max}(x)} f(x,y)dy \\
	{\bf \text{Finally, } \int \int_R f(x,y) dA = \int_{x_{min}}^{x_{max}} \left[ \int_{y_{min}(x)}^{y_{max}(x)} f(x,y) dy \right] dx}
\end{align*}
This is called an \textbf{Iterated Integral} because one integral is integrated over the next.
BTW: $dA$ becomes $dy \cdot dx$ because $dA$ is the area of each infinitesimal rectangle which equals $dy \cdot dx$
\newline Left off \href{https://youtu.be/YP_B0AapU0c?t=2148}{here}

\end{document}
