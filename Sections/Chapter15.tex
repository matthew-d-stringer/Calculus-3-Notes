\documentclass[../main.tex]{subfiles}
% !TEX root = ../main.tex

\begin{document}
\section{Chapter 15 Double Integral}

\subsection{Definition of the Double Integral}
Double Integral = volume below the graph $z = f(x,y)$ over a region $R$ in xy-plane.
\begin{equation*}
	\iint_R f(x,y)dA
\end{equation*}
\textbf{Definition: } Cut R into small pieces of area $\Delta A$
Lets divide the region $R$ into $i$ x and y components
The height of each rectangular prism is $f(x_i, y_i)$ and the area of the region underneath is $\Delta A_i$
\begin{equation*}
	\sum_i f(x_i,y_i) \Delta A_i 
\end{equation*}
Finally, you get take the limit as $\Delta A_i \to 0$ to get the double integral.

\subsection{Calculating the Double Integral}
To compute the double integral, take \textbf{slices}. 
Let $S(x)$ = area of slice by plane parallel to the yz-plane.
Then, 
\begin{align*}
	\text{volume} = \int_{x_{min}}^{x_{max}} S(x)dx. \\
	S(x) = \int_{y_{min}(x)}^{y_{max}(x)} f(x,y)dy \\
	{\bf \text{Finally, } \iint_R f(x,y) dA = \int_{x_{min}}^{x_{max}} \left[ \int_{y_{min}(x)}^{y_{max}(x)} f(x,y) dy \right] dx}
\end{align*}
This is called an \textbf{Iterated Integral} because one integral is integrated over the next.
BTW: $dA$ becomes $dy \cdot dx$ because $dA$ is the area of each infinitesimal rectangle which equals $dy \cdot dx$

\subsection{Double Integral in Polar Coordinate}
Slices of circles on the polar coordinate plane are approximately rectangles when they are very tiny. Therefore,
\begin{equation*}
	\Delta A \approx \Delta r \Delta \theta
\end{equation*}
and therefore, 
\begin{equation*}
	dA = dr d\theta
\end{equation*}
If you have the function $f(\theta,r)$, then in order to find the volume under the curve, use the formula
\begin{equation*}
	V = \iint_R rf(\theta,r) dA
\end{equation*}

\subsection{Applications of Double Integrals}
\begin{enumerate}
	\item Find area of a region $R$
	\begin{equation*}
		A = \iint_R 1 dA 
	\end{equation*}
	\item Finding mass of a flat object with density $\delta = $mass per unit area
	\begin{align*}
		\Delta m = \delta \Delta A \\
		m = \iint_R \delta dA
	\end{align*}
	\item Finding average value of $f$ in region $R$
	\begin{align*}
		\bar f = \frac1{\text{Area}} \iint_R f dA
	\end{align*}
	Weighted average of $f$ with density $\delta$: 
	\begin{equation*}
		\frac1{\text{Mass}(R)} \iint_R f \delta dA
	\end{equation*}
	\item Center of Mass of a (planar) object (with density $\delta$)? \newline
		Center of mass is at $(\bar x, \bar y)$ where 
		\begin{align*}
			\bar x = \frac1{\text{Mass}} \iint_R x \delta dA \\
			\bar y = \frac1{\text{Mass}} \iint_R y \delta dA \\
		\end{align*}
	\item Moment of inertia of an object about the origin
		\begin{equation*}
			\iint_R r^2 \delta \; dA = \iint_R (x^2 + y^2) \delta \; dA
		\end{equation*}
\end{enumerate}

\href{https://youtu.be/60e4hdCi1D4?t=1308}{\bf Left off here}

\subsection{Triple Integrals}
\begin{itemize}
    \item Really similar to double integrals
    \item Finds the summation of volumes of a 3 dimensional region
\end{itemize}
\begin{equation*}
    \iiint_R f \; dV
\end{equation*}

\subsection{Applications of Triple Integrals}
\begin{itemize}
    \item Mass
        \begin{equation*}
            \text{Mass} = \iint_R \delta \; dV 
        \end{equation*}
    \item Average Value 
        \begin{equation*}
            \bar f = \frac1{\text{Vol}(R)} \iiint_R f \; dV
        \end{equation*}
        or with density: 
        \begin{equation*}
            \bar f = \frac1{\text{Mass}(R)} \iiint_R f \delta \; dV
        \end{equation*}
    \item Center of Mass: $(\bar x, \bar y, \bar z)$
        \begin{align*}
            \bar x = \frac1{\text{Mass}(R)} \iiint_R x \delta \; dV \\
            \bar y = \frac1{\text{Mass}(R)} \iiint_R y \delta \; dV \\
            \bar z = \frac1{\text{Mass}(R)} \iiint_R z \delta \; dV
        \end{align*}
    \item Moment of Inertia: 
        \begin{align*}
            \iiint_R (\text{distance to axis})^2 \delta \; dV \\
			I_z = \iiint_R r^2 \delta \; dV = \iiint_R \delta (x^2 + y^2) \; dV \\
			I_y = \iiint_R \delta (x^2 + z^2) \; dV \\
			I_x = \iiint_R \delta (y^2 + z^2) \; dV \\
        \end{align*}
\end{itemize}

\end{document}