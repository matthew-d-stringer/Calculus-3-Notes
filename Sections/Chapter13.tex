\documentclass[../main.tex]{subfiles}
% !TEX root = ../main.tex

\begin{document}

\section{Section 13.2}

\subsection{Ideal Projectile Motion}
If $v_0$ makes an angle $\alpha$ with the horizon, then 
\begin{equation*}
v_0 = (|v_0|\cos \alpha){\ihat} + (|v_0|\sin \alpha){\jhat}
\end{equation*}
Also, assume $r_0 = 0{\ihat} + 0{\hat \j}$. With these formulas it can be derived that
\begin{equation*}
r(t) = -\frac12 gt^2 {\hat \j} + v_0t
\end{equation*}

\subsubsection{Range of a Projectile}
\begin{equation*}
R = \frac{v_0^2}{g} \sin (2\alpha)
\end{equation*}
Where $R$ is the range, $v_0$ is the initial velocity, $g$ is acceleration due to gravity, and 
$\alpha$ is the launch angle

\subsubsection{Maximum Height of a Projectile}
Lets say that $y_{max}$ is the maximum height, then:
\begin{equation*}
	y_{max} = \frac{v_0 \sin \alpha}{2g}
\end{equation*}

\section{Section 13.3}

\subsection{Finding Tangent Vector to a Path}
If $r(t)$ is the function for the path, then $r'(t)$ represents the tangent vector for the path.

\subsection{Length of Path}
If $r(t)$ is the function for the path and $D$ represents the length of the path from $t \in (t_0, t_1)$,
then 
\begin{equation*}
	D = \int_{t_0}^{t_1} |r'(t)|dt
\end{equation*}

\section{Section 13.4}

\subsection{Curvature}
If $T$ is the tangent unit vector of a smooth curve, then the curvature of the curve is
\begin{equation*}
\kappa = \left| \frac{dT}{ds} \right|
\end{equation*}
where s is arc length.

\subsection{Formula for Calculating Curvature}
If $s(t)$ is a smooth curve, then curvature is 
\begin{equation*}
%\kappa = \frac1{|s'(t)|} \cdot \left|\frac{dT}{dt}\right|
\kappa = \left| \frac{\frac{dT}{dt}}{\frac{ds}{dt}} \right|
\end{equation*}
where 
\begin{equation*}
T = \frac{s'(t)}{|s'(t)|}
\end{equation*}
and $v = r'(t)$

\subsection{Principal Unit Normal}
At any point $\kappa \neq 0$, the principal unit normal vector for a curve is 
\begin{equation*}
N = \frac{1}{\kappa} \frac{dT}{ds}
\end{equation*}

\subsection{Formula for Calculating N}
If $r(t)$ is a smooth curve, then the principal unit normal is
\begin{equation*}
N = \frac{\frac{dT}{dt}}{\left| \frac{dT}{dt} \right |}
\end{equation*}

\section{Section 13.5}

\subsection{Binormal Vector}
The binormal vector ($B$) is a vector that is orthogonal to both $T$ and $N$. 
\begin{equation*}
B = T \times N 
\end{equation*}

\subsection{Acceleration Vector}
If the acceleration vector is 
\begin{equation*}
a = a_T T + a_N N
\end{equation*}
then 
\begin{equation*}
a_T = s''(t) = \frac{d}{dt} |s'(t)| \text{ and } a_N = \kappa \left( \frac{ds}{dt} \right)^2 = \kappa |s'(t)|^2
\end{equation*}
also
\begin{equation*}
a_N = \sqrt{|s''(t)|^2 - a_T^2}
\end{equation*}

\subsection{Torsion}
\begin{equation*}
\frac{dB}{ds} = \frac{d(T \times N)}{ds} = \frac{dT}{ds} \times N + T \times \frac{dN}{ds}
\end{equation*}
Since $N$ has the same direction as $\frac{dT}{ds}$, $N \times \frac{dT}{ds} = 0$
\begin{equation*}
\frac{dB}{ds} = T \times \frac{dN}{ds}
\end{equation*}
Because $\frac{dB}{ds}$ is orthogonal to $B$ and $T$, it must be parallel to $N$. Therefore, 
$\frac{dB}{ds}$ is a scalar multiple of $N$ and 
\begin{equation*}
\frac{dB}{ds} = -\tau N
\end{equation*}
where $\tau$ is called the \textit{torsion} of the curve. Note that 
\begin{equation*}
\frac{dB}{ds} \cdot N = -\tau N \cdot N = -\tau
\end{equation*}

\subsection{Calculating Torsion}
If $B = T \times N$, the torsion of a smooth curve is 
\begin{equation*}
\tau = - \frac{dB}{ds} \cdot N
\end{equation*}
also, 
\begin{equation*}
\tau = \frac{
	\begin{vmatrix}
	\dot x & \dot y & \dot z \\
	\ddot x & \ddot y & \ddot z \\
	\dddot x & \dddot y & \dddot z \\
	\end{vmatrix}
}{| v \times a |^2}
\end{equation*}

\section{Section 13.6}
\subsection{Motion in Polar and Cylindrical Coordinates}
When a particle at $P(r,\theta)$ moves along a curve in the polar coordinate plane, we express its
position in terms of moving unit vectors:
\begin{equation*}
u_r = (\cos \theta) \ihat + (\sin \theta) \jhat \quad u_\theta = -(\sin \theta)\ihat + (\cos \theta)\jhat
\end{equation*}
where $u_r$ is tangent to the vector $\begin{bmatrix} \theta \\ r \end{bmatrix}$ and $u_\theta$ is 
normal to it. When we differentiate $u_r$ and $u_\theta$ with respect to $t$, we can see how they 
change with time.
\begin{equation*}
\dot u_r = (-\dot{\theta}\sin \theta) \ihat + (\dot{\theta}\cos \theta) \jhat \quad 
u_\theta = -(\dot{\theta}\sin \theta) \ihat - (\dot{\theta}\sin \theta) \jhat
\end{equation*}
which equals 
\begin{equation*}
\dot u_r = \dot{\theta} u_\theta \quad u_\theta = -\dot{\theta} u_r
\end{equation*}

\subsection{Polar Velocity Vector}
\begin{equation*}
v = \dot r = \frac{d}{dt} \left( r u_r \right) = \dot r u_r + r \dot u_r = \dot r u_r + r \dot \theta u_\theta
\end{equation*}

\subsection{Polar Acceleration Vector}
\begin{equation*}
a = \dot v = (\ddot r u_r + \dot r \dot u_r) + 
(\dot r \dot \theta u_\theta + r \ddot \theta u_\theta + r \dot \theta \dot u_\theta)
\end{equation*}
which becomes 
\begin{equation*}
a = (\ddot r - r \dot \theta^2)u_r + (r \ddot \theta + 2 \dot r \dot \theta) u_\theta
\end{equation*}

\subsection{Planet Movement}
If $r$ is the radius vector from the center of a sun of mass $M$ to the center of a planet of mass $m$,
then the force $F$ of the gravitational attraction between the planet and the sun is:
\begin{equation*}
F = - \frac{GmM}{|r|^2} \frac{r}{|r|}
\end{equation*}
where $G$ is the universal gravitational constant. 
\begin{equation*}
G = 6.6726 \times 10^{-11} \si{\newton \meter^2 \kilogram^{-2}}
\end{equation*}
By using Newton's second law, we can find
\begin{equation*}
\ddot r = - \frac{GM}{|r|^2} \frac{r}{|r|}
\end{equation*}
Because $\ddot r$ is a scalar multiple of $r$, we know 
\begin{equation*}
r \times \ddot r = 0
\end{equation*}
so we know
\begin{equation*}
\frac{d}{dt} (r \times \dot r) = \dot r \times \dot r + r \times \ddot r = r \times \ddot r = 0
\end{equation*}
meaning that the cross between $r$ and $\dot r$ is constant
\begin{equation*}
r \times \dot r = C
\end{equation*}

\subsection{Kepler's First Law}
The eccentricity of the ellipse that the planet follows is 
\begin{equation*}
e = \frac{r_0 v_0^2}{GM} - 1
\end{equation*}
and the polar equation is 
\begin{equation*}
r = \frac{(1+e)r_0}{1+e \cos \theta}
\end{equation*}
where $r_0$ is the minimum distance from the sun. The sun's mass $M$ is 
$1.99 \times 10^{30} \si{ \kilogram}$

\subsection{Kepler's Third Law}
The time $T$ it takes a planet to go around its sun once is the planet's orbital period. Kepler's
Third Law says that and the orbit's semimajor axis $a$ are related by
\begin{equation*}
\frac{T^2}{a^3} = \frac{4\pi^2}{GM}
\end{equation*}

\end{document}